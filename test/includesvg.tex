\listfiles
\documentclass[parskip=full,ngerman]{scrreprt}% Test
\usepackage{selinput}\SelectInputMappings{adieresis={ä},germandbls={ß}}
\usepackage[T1]{fontenc}
\usepackage{babel}
\usepackage[usexcolor=true,usetransparent=true,draft=false]{svg}
\usepackage[extract=true,clean=false]{svg-extract}

%\usepackage{etoolbox}

%\svgsetup{lastpage=false,convertformat=,convert=gs}

%\svghidepreamblestart
%\usepackage{showframe}

%\usepackage{blindtext,subfig}

%\makeatletter
%\setlength{\@fptop}{0pt}
%\setlength{\@fpsep}{0pt}% twocolumn
%\makeatother
%
\pdfsuppresswarningpagegroup=1%
%
%\parskip=0pt
%\parindent=0pt

%\svghidepreambleend

%\usepackage{amsmath,etoolbox}

%\RequirePackage{luatex85}[2016/06/15]% standalone
\begin{document}
\svgpath{{./svg/}}
\begin{figure}
  \begin{center}
%  \includegraphics{\dq svg-extract/svg-example_svg-tex-extract\dq}
%  \subfloat[This text is too large!]{%
      \includesvg[%
        extractformat={pdf,eps,ps},clean=false,%convert,%
        extractpreamble={bla.tex},%latexopt=--shell-escape,
%        inkscapearea=page,%
%        extr
%        name=Fig.1a,preamble=examples/preamble.tex,pdf,width=5cm,eps,%
      ]{./svg/svg-example}%
%  }
%  \subfloat[This text is too large!]{%
%      \newincludesvg[%
%%        name=Fig.1b,preamble=examples/preamble.tex,%
%%        eps,pretex=\small,width=5cm%
%      ]{svg-example}%
%  }
%    \caption{An example figure.}
  \end{center}
\end{figure}

\chapter{blubb}
\chapter{blubb}%\label{fig:example}

\end{document}

\svgpath{{svg-test/}{"C:/Users/Hanisch/Documents/Dateien/svg"}}

%\begin{equation}
%a^2+b^2=c^2\label{eq:emc1}
%\end{equation}
\begin{equation}
a^2+b^2=c^2\label{eq:emc2}
\end{equation}

\svgsetup{pretex=\color{red}}

%\svgsetup{lastpage=false}

\fboxsep=-\fboxrule
%\begin{figure}
%\fbox{\includegraphics[width=\textwidth,angle=10]{svg-out/Seriell_svg-raw.png}}

%\end{figure}

%\def\svgwidth{400pt}
%\def\svgscale{0.6}

\begin{figure}
%\def\svgwidth{\textwidth}
%\def\svgscale{0.8}
\fbox{\newincludesvg[%
%  inkscapelatex=false,
%  inkscapepath=basesubdir,%
%  exclude=true,
%  angle=10,%origin=c,
%  svgpath=C:/Users/Hanisch/Documents/Dateien/svg,
%  extractpreamble=test.aaa,
  clean=false,extractformat={pdf,eps},convertformat=eps,
%  extractpreamble=ssss.tex,
%  inkscape=raw,
%  inkscape=eps,
%  convertdensity=600x300,
%  convertdensity=555x300x,
%  angle=10,
%  inkscapedpi=,
%  outname="bla",
%  extractname=hamster,
%  lastpage=false,
%  draft,
%  extract=false,%
%  convertdpi=400dpi,
%  convertdpi={png+=700x300dpi},
%  convertdpi={jpg+=333dpix666dpi},
%  convert=true,%
%  lastpage=3,%
%  lastpage=false,%
%  convert=gs64,%
%  extractformat=,%{pdf,eps,ps},%
%  convertformat={png,tiff,jpg,eps},%
%  gsdevice={eps=\relax},
%  gsopt=bla,%
%%  gsopt={png=foo},%
%  gsopt={png+=bar},%
%  magicksetting=-depth 4,
%  magicksetting={png=-depth 8},
%  magickoperator=-resize 25\%,
%  magickoperator={png=-resize 25\%},
%  convertformat=png,
%  convertdensity=,%
  width=300pt,%
%  extractwidth=\relax,
%  extractwidth=400pt,
%  scale=1.1,
%  extractscale=1.2,
%  extractpretex=inherit,
%  extractwidth=,
%  clean=false,
%  angle=-180,
%  draft,
%  height=4cm,
%  scale=0.5pt,
%  pretex=aaa,
%  apptex=bbb,
]{Seriell}%
}%
\caption{aaa}
\end{figure}

%\begin{figure}
%%\newincludesvg[%
%%  width=200pt,%
%%%  path="svgtest",
%%%  extractname=bla blau,
%%%  angle=30,%
%%%  svgpath=svg test,%
%%]{testimage}
%\newincludesvg[%
%%  draft,
%%  extractname=bla,
%  width=.2\textwidth,%
%%  extractformat={eps},%
%%  angle=30,%
%%  clean=false,
%%  inkscape=raw,
%%  svgpath={"svg test/"},%
%]{newimage}
%\end{figure}


\end{document}
